\documentclass{article}
\usepackage{color} 
\usepackage{listings} 
\usepackage{caption}
\DeclareCaptionFont{white}{\color{black}} 
\DeclareCaptionFormat{listing}{\colorbox{gray}{\parbox{\textwidth}{#1#2#3}}}
\captionsetup[lstlisting]{format=listing,labelfont=white,textfont=white}

\usepackage[utf8]{inputenc}
\usepackage{authblk}
\usepackage[dvipsnames]{xcolor}
\usepackage[T2A]{fontenc}
\usepackage[russian]{babel}

\title{Code Style}
\author{Лещенко Леонид, ИА-032;}
\affil{СибГУТИ, email: kallygary@yandex.ru}
\date{Февраль 2022}

\begin{document}



\lstset{ 
language=C,                 
basicstyle=\small\sffamily, 
numbers=left,               
numberstyle=\tiny,           
stepnumber=1,                   
numbersep=0pt,                
backgroundcolor=\color{white}, 
showspaces=false,            
showstringspaces=false,      
showtabs=false,            
frame=single,              
tabsize=2,                 
captionpos=t,              
breaklines=true,           
breakatwhitespace=false, 
escapeinside={\%*}{*)}   
}

\maketitle

\newpage
\tableofcontents

\newpage
\section{Вступление}
Созадние Код-Стайла для C++

\section{С++}
Мы используем С++ версии 14

\newpage
  \begin{center}
    \section*{Код-Стайл для С++}
  \end{center}
  
    \setcounter{section}{0}
    \section{Пробелы и отступы}
  
        \subsection{Отделяйте пробелами фигурные скобки:} 
            \begin{lstlisting}
    if (a == b) {
        foo();
    }
            \end{lstlisting}
    
        \subsection{Пробелы ставятся между операторами и\\ операндами:} 
            \begin{lstlisting}
    int x = (a + b) * c / d + foo(); 
            \end{lstlisting}
    
        \subsection{Оставляйте пустые линии между функциями и\\ между группами выражений:} 
\begin{lstlisting}
    void foo() {
     ...
    }
                   
    void bar() {
     ...
    }
    \end{lstlisting}
    
    \section{Названия и переменные}
        \subsection{Если определенная константа часто используется в вашем коде, то обозначьте её как const и всегда \\ссылайтесь на данную константу}
            \begin{lstlisting}
    const int VOTING_AGE = 18;
            \end{lstlisting}
            
\newpage
        \subsection{Никогда не объявляйте изменяемую глобальную\\ переменную}
            \begin{lstlisting}
    int func1() {
        return 42;
    }

    void func2(int& count) {
        count++;
    }

    int main() {
        int count = func1();
        func2(count);
    }
            \end{lstlisting}
  
    \section{Базовые выражения С++}
        \subsection{Используйте для вывода текста оператор cout\\ вместо printf }
            \begin{lstlisting}
    cout << "Hello, world!" << endl;
            \end{lstlisting}
            
        \subsection{Когда используете операторы управления вроде if/else, for, while, всегда используйте {} и соответствующие отступы, даже если тело всего оператора \\управления состоит лишь из одной строки:}
            \begin{lstlisting}
    if (size == 0) {
    return;
    } else {
    for (int i = 0; i < 10; i++) {
        cout << "ok" << endl;
        }
    }
            \end{lstlisting}
            
        \subsection{Если у вас есть выражение if / else, которое\\ возвращает логическое значение, возвращайте результаты теста напрямую:}
            \begin{lstlisting}
    return score1 == score2;
            \end{lstlisting}
            
            
\newpage
        \subsection{Никогда не проверяйте значения логического типа, используя == или != с true или false:}
            \begin{lstlisting}
    if (x) {
    ...
    } else {
    ...
    }
            \end{lstlisting}
            
\newpage
    \begin{center}
        \section*{Вывод}
        \addcontentsline{toc}{section}{Вывод}
    \end{center} 
    
    Мы разобрали основные понятия и правила связанные с Код-стайлом С++ и составили отчет с помощью LaTex

\newpage

    \begin{center}
        \section*{Список литературы}
        \addcontentsline{toc}{section}{Список литературы}
    \end{center} 
    
\begin{thebibliography}{7}
    \bibitem{1} https://habr.com/ru/post/172091/
    \bibitem{2} https://google.github.io/styleguide/cppguide.html
    \bibitem{3} https://gist.github.com/azoyan/b545f7b926f1f7fb40f8c285e3f5c545
\end{thebibliography}

        
\end{document}